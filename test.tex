
\nonstopmode

%% LyX 2.3.2 created this file.  For more info, see http://www.lyx.org/.
%% Do not edit unless you really know what you are doing.
\documentclass[12pt,british]{article}
\renewcommand{\familydefault}{\sfdefault}
\usepackage[T1]{fontenc}
\usepackage[latin9]{inputenc}
\usepackage[landscape]{geometry}
\geometry{verbose,tmargin=3cm,bmargin=3cm,lmargin=4cm,rmargin=4cm}
\setlength{\parskip}{\medskipamount}
\setlength{\parindent}{0pt}
\usepackage{float}
\PassOptionsToPackage{normalem}{ulem}
\usepackage{ulem}

\makeatletter
%%%%%%%%%%%%%%%%%%%%%%%%%%%%%% User specified LaTeX commands.

\usepackage{babel}
\usepackage{graphicx}
\usepackage{epstopdf}
\usepackage{placeins}
\usepackage{listings}
\usepackage{lmodern}

% Alter some LaTeX defaults for better treatment of figures:
% See p.105 of "TeX Unbound" for suggested values.
% See pp. 199-200 of Lamport's "LaTeX" book for details.
%   General parameters, for ALL pages:
\renewcommand{\topfraction}{0.9}	% max fraction of floats at top
\renewcommand{\bottomfraction}{0.8}	% max fraction of floats at bottom
%   Parameters for TEXT pages (not float pages):
\setcounter{topnumber}{2}
\setcounter{bottomnumber}{2}
\setcounter{totalnumber}{4}     % 2 may work better
\renewcommand{\textfraction}{0.01}	% allow minimal text w. figs
%   Parameters for FLOAT pages (not text pages):
\renewcommand{\floatpagefraction}{0.5}	% require fuller float pages
% N.B.: floatpagefraction MUST be less than topfraction !!



\usepackage[outermarks]{titleps} 

% let's assume top/bot
\def\presectiontitle{}
\newpagestyle{main}{
   
\sethead
{} 
{    \Large \bf
      \bottitlemarks
      \ifx\sectiontitle\presectiontitle             
      \sectiontitle  
      \ (cont.)
      \fi            
      \bottitlemarks           
      \global\let\presectiontitle\sectiontitle 
}
{}

\setfoot
{ \thepage }
{}  
{ \ Magic of Makefiles --  -- Guy Nankivell}
\footrule	
\headrule
}


\let\stdsection\section
\renewcommand\section{\newpage\stdsection}

\AtBeginDocument{
  \def\labelitemiii{\(\circ\)}
}

\makeatother


\begin{document}
\date{}
\title{Magic of Makefiles}
\author{Guy Nankivell}

\maketitle
\newpage{}

\LARGE
\pagestyle{main}

\section{History of Makefiles}
\begin{itemize}
	\item Born in 1976 (43 years ago)
\end{itemize}

\section{Requirements of a Build System}
\begin{itemize}
	\item Portable
	\item Efficient (Stop unecessary recompilations)
	\item Language Agnostic (nested projects)
	\item Fast
	\item Modular
	\item Flexible
\end{itemize}

\section{A Little Note...}
What I henceforth refer to as `make' is
the GNU implementation. This is due to being the most common reference
for \texttt{make} in the wild.



\section{Cornerstone Features of Make}
\begin{itemize}
\item Parallel
\end{itemize}


\section{Okay, but can it work for us?}
It is used as the build system for the Linux kernel so it must be
doing something right...


\section{Drawbacks}
\begin{itemize}
\item Arcane Syntax
\end{itemize}


\section{Compiling this presentation...}
\lstset{basicstyle=\scriptsize\ttfamily,language=[gnu] make}
\begin{lstlisting}
CC =            pdflatex
SRC =           talk.tex
OUT_DIR =       tmp
CFLAGS =        -output-directory $(OUT_DIR)
FILE_REDIR =    /dev/null
FNAME =         out.pdf

all: $(OUT_DIR)
@ $(CC) $(CFLAGS) $(SRC) > $(FILE_REDIR)
@ mv $(OUT_DIR)/*.pdf ./$(FNAME)

clean:
@ $(RM) -r $(OUT_DIR)/

$(OUT_DIR): 
@ mkdir -p $(OUT_DIR)/

\end{lstlisting}

\section{Plays nicely with \texttt{vim}}
From command mode in \texttt{vim} all one has to do to action the
\texttt{Makefile} in the same directory to which your session is running
is;
\begin{center}
	\texttt{:make}
\end{center}

Which keeps your editing session alive and allows you to view the
output of the command and then you need only press enter to drop back into
editing mode

\section{\textit{Finis.}}

\begin{center}
			\texttt{Questions?}
\end{center}


\end{document}
